%!TEX root = ../main.tex
\section{Introduction}

Designing a control system for a dc motor is a multi-step process. It involves the parametrization of the system, the design of the controller as well as simulations and experiments to determine what is the best strategy to follow in certain situations depending on the plant. The purpose of this dc report is to go through these steps and provide the reader with the information required in such a procedure. Also, this report has been written using the citation~\cite{feedback} as the main source of theoretical knowledge. The motor model that is used is the Pittman 9234S007-R1~\cite{pittmann}.

Finding the parameters such as the inductance and the constants of a motor can be achieved by experimental procedures in a laboratory environment. In many cases, more than one experiment are done as well as more than one methods have been followed for greater accuracy of the data. These parameters then are used from different controllers to drive and test the motor. Designing a controller requires analysis of different topologies such as the PID and IPD as well as the required order needed to control the motor effectively. Apart from that, implementing different strategies to improve the behaviour of the overall system such as noise filtering and anti-windup design plays an important role in the process of designing a controller. The effectiveness of these strategies are not only affected by the environment and the overall system, but also from the settling time, which is a very important variable to take into account.
\todo{Write about the experiments and simulations}