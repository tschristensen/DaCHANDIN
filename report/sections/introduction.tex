%!TEX root = ../main.tex
\section{Introduction}
Designing a control system for a DC motor is a multi-step process. 
It involves the parametrization of the system, the design of the controller as well as simulations and experiments to determine the best strategy to follow in certain situations depending on the plant. 
The purpose of this report is to go through these steps and provide the reader with the information required in such a procedure. 
Also, this report has been written using the citation \cite{feedback} as the main source of theoretical knowledge. 
The motor models that are used are the Pittman 9234S007-R1 and 9234S006 \cite{pittmann}.
\\

Finding parameters such as the inductance and constants of a motor can be achieved by experimental procedures using appropriate instruments. 
The parameters to be calculated throughout this report are; the voltage constant, the armature resistance, the viscous damping factor, the motor inductance and the inertia. 
In some cases, more than one method is followed for greater accuracy of the data. 
The data is compared with the datasheet of the motors to validate the results. 
The parameters found by experimentation are used in different controllers to drive and test the motor. 
\\

Throughout this report different controller topologies will be utilised, such as a PID and an IPD controller. 
Additionally, the order required to control the motor effectively will be determined. 
The transfer functions will be analysed in order to better understand how they function.
Some work will be done to implement different strategies to improve the behaviour of the overall system, such as noise filtering and anti-windup. 
The design of such systems play an important role in the process of designing a controller. 
The effectiveness of these strategies are not only affected by the environment and the overall system, but also from the settling time, which is an important variable to take into account. 
Lastly, the parameters are inserted in the gain equations of the controllers to receive the gains for a specified system.
\\

After designing a controller, simulations must be run in order to examine how the system behaves under ideal conditions. 
Different controllers will be examined such that the most efficient and suitable one can be chosen for a specific purpose. 
Specifically, the IPD and PI are tested in this report with and without load or extra inertia at different speeds. 
The two controllers are also tested under real conditions using the Pittman motors with the same parameters such as speed and load step that are used under the simulations. 
Finally, the results from the real world and the simulated experiments are compared to each other to observe any similarities or differences at the overall system's behaviour.
