%!TEX root = ../main.tex
\section{Introduction}

Designing a control system for a DC motor is a multi-step process. It involves the parametrization of the system, the design of the controller as well as simulations and experiments to determine what is the best strategy to follow in certain situations depending on the plant. The purpose of this DC report is to go through these steps and provide the reader with the information required in such a procedure. Also, this report has been written using the citation~\cite{feedback} as the main source of theoretical knowledge. The motor models that are used are the Pittman 9234S007-R1 and 9234S006~\cite{pittmann}.
\\

Finding the parameters such as the inductance and the constants of a motor can be achieved by experimental procedures in a laboratory environment using the appropriate instruments. The parameters that required to be calculated are the voltage constant, the armature resistance, the viscous damping factor and the motor inductance and inertia. In many cases, more than one experiment are done as well as more than one methods are followed for greater accuracy of the data. Then, the data is compared with the datasheet of the motors to see if the values coincide or if they deviate. The parameters found by the experiments are used as input to different controllers to drive and test the motor. 
\\

Designing a controller requires analysis of different topologies such as the PID and IPD as well as the required order needed to control the motor effectively. Also, their transfer functions need to be analysed and comprehended in order to better understand how they function. Apart from that, implementing different strategies to improve the behaviour of the overall system such as noise filtering and anti-windup design plays an important role in the process of designing a controller. The effectiveness of these strategies are not only affected by the environment and the overall system, but also from the settling time, which is a very important variable to take into account. Lastly, the parameters are inserted in the gain equations of the controllers to receive the gains for a specified system.
\\

After designing a controller, simulations must be run in order to examine how the system is expected to behave in various circumstances and under ideal conditions. Different controllers are being examined so the most efficient and suitable one can be chosen for a specific purpose. Specifically, the IPD and PI are tested under this report with and without load or extra inertia at different speeds. The two controllers are also tested under real conditions using the Pittman motors with the same parameters such as speed and load step that are used under the simulations. Finally, the results from the real world and the simulated experiments are compared to each other to observe any similarities or differences at the overall system's behaviour.

\todo{make the sections go to the next page)}
\todo{fix the pages, there are some spaces towards the end. Large spaces!}