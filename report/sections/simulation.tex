\section{Simulation}
Before the controllers can be tested on the real system a simulation must be done. The controllers that have been chosen are the IPD and PI, where PI is the PID controller with $K_D$ set to 0 Speed control is simulated at two different angular velocities, 200~rad/s and~125 rad/s To prevent the saturation limit to be reached, the settling time is set to 0.1~ms\todo{why settling time 0.1 and not 0.01}.

\begin{figure}[h!]
	\centering
	\begin{subfigure}[b]{0.45\textwidth}
		\includegraphics[width=\textwidth]{graphics/PI_single125}
		\caption{PI Speed control at 125 rad/s}
		\label{fig:pisingle125}
	\end{subfigure}
	~ %add desired spacing between images, e. g. ~, \quad, \qquad, \hfill etc. 
	%(or a blank line to force the subfigure onto a new line)
	\begin{subfigure}[b]{0.45\textwidth}
		\includegraphics[width=\textwidth]{graphics/PI_single200}
		\caption{PI Speed control at 200 rad/s}
		\label{fig:pisingle200}
	\end{subfigure}
	\caption{The PI controller at two different angular velocities. it is noticed that the desired settling time is reached in both cases.}\label{fig:pisingle}
\end{figure}



\begin{figure}[h!]
	\centering
	\begin{subfigure}[b]{0.45\textwidth}
		\includegraphics[width=\textwidth]{graphics/PI_load125}
		\caption{PI Speed control at 125 rad/s}
		\label{fig:piload125}
	\end{subfigure}
	~ %add desired spacing between images, e. g. ~, \quad, \qquad, \hfill etc. 
	%(or a blank line to force the subfigure onto a new line)
	\begin{subfigure}[b]{0.45\textwidth}
		\includegraphics[width=\textwidth]{graphics/PI_load200}
		\caption{PI Speed control at 200 rad/s}
		\label{fig:piload200}
	\end{subfigure}
	\caption{The PI controller at two different angular velocities with added load. Desired settling time is not reached. Instead it introduces a overshoot and does not become stable until close to 1s}\label{fig:piload}
\end{figure}


\begin{figure}[h!]
	\centering
	\begin{subfigure}[b]{0.45\textwidth}
		\includegraphics[width=\textwidth]{graphics/IPD_single125}
		\caption{IPD Speed control at 125 rad/s}
		\label{fig:ipdsingle125}
	\end{subfigure}
	~ %add desired spacing between images, e. g. ~, \quad, \qquad, \hfill etc. 
	%(or a blank line to force the subfigure onto a new line)
	\begin{subfigure}[b]{0.45\textwidth}
		\includegraphics[width=\textwidth]{graphics/IPD_single200}
		\caption{IPD Speed control at 200 rad/s}
		\label{fig:ipdsingle200}
	\end{subfigure}
	\caption{The PI controller at two different angular velocities. A delay is introduced. The desired settling time is reached from when the controller responses.}\label{fig:ipdsingle}
\end{figure}

\begin{figure}[h!]
	\centering
	\begin{subfigure}[b]{0.45\textwidth}
		\includegraphics[width=\textwidth]{graphics/IPD_load125}
		\caption{IPD Speed control at 125 rad/s}
		\label{fig:ipdload125}
	\end{subfigure}
	~ %add desired spacing between images, e. g. ~, \quad, \qquad, \hfill etc. 
	%(or a blank line to force the subfigure onto a new line)
	\begin{subfigure}[b]{0.45\textwidth}
		\includegraphics[width=\textwidth]{graphics/IPD_load200}
		\caption{IPD Speed control at 200 rad/s}
		\label{fig:ipdload200}
	\end{subfigure}
	\caption{The PI controller at two different angular velocities with added load. The desired settling time is far from being reached. The system becomes stable at around 12s}\label{fig:ipdload}
\end{figure}

The simulation confirms that the controllers behave similar for both angular velocities as seen in figures~\ref{fig:pisingle} and~\ref{fig:ipdsingle}. However when load is added, the controllers seem to have a hard time on settling, Figure~\ref{fig:piload} and~\ref{fig:ipdload}. this result was expected thus the controllers were designed with only single motor in mind. Added load changes the system behavior and should be controlled with different controller gains.\todo{why delay??}

