% !TeX root = ../main.tex
\section{Conclusion}

The report has gone through the steps required to design a control system for a DC motor and various conclusions have been drawn. The parameters from a motor's datasheet may not always coincide with the ones found by experimenting. In the case of the specific motor that the experiments were run, the inductance and the armature resistance are out of specification, 17.7\% and 25\% respectively, while the motor inertia deviates as well by an outstanding 45.4\%. 
\\

During the design of the PID and IPD controllers, one of the strategies that are implemented in order to make them more stable exhibits some differences in behaviour and efficiency, depending on the system. Specifically, the gain of the anti-windup design for the two controllers is completely different. A suitable gain for the PID is a value of 10, while for the IPD, 10000 is a far better value. On the other hand, the noise filtering is implemented with no differences to the various systems. Furthermore, the settling time suffers for very low values such as $T_s=0.01$ due to reaching the saturation of the system, which is counteracted by the anti-windup strategy, although not completely. Lastly, the characteristic equations of the two different controllers are the same, which contributes positively to the overall analysis.
\\

The simulations showed that for $T_s=0.1$ both controllers manage to reach their target within the specified settling time. However, that is not the case when a step load is introduced into the system. The PI delays almost $1s$ to settle again introducing an overshoot as well, but with no further difficulties. The IPD on the other hand finds it much more difficult to stabilize again. It takes much longer to achieve stability, requiring around $12s$. The real-world experiments showed some differences with the simulations' results. During the velocity step experiment, the PI settles within the settling time with an overshoot not present in the simulation, while the load step behaviour was observed to be the same in both the experiment and the simulation.
\todo{Write some more, after the last section about the performance is cleared out (PID or IPD????)}