%!TEX root = ../main.tex

%\documentclass{standalone}

%\usepackage{tikz}

% The block diagram code is probably more verbose than necessary
\begin{tikzpicture}[auto, node distance=3.5cm,>=latex']
    % We start by placing the blocks
    \node [block] (motor) {Motors};
    \node [block, left of=motor] (hbro) {H-bridge};
    \node [block, left of=hbro] (dspace) {dSPACE};
    \node [block, right of=motor] (encoder) {Encoders};
    \node [below= 1cm of motor] (point) {};
    \node [above= 1cm of motor] (point1) {};
    \node [blockg, right =0.5cm of point1] (scope) {Oscilloscope};
    \node [blockg, left =0.5cm of point1] (multi) {Multimeter};


    \draw[->] (hbro) -- (motor);
    \draw[->] (dspace) -- (hbro);
    \draw[->] (motor) -- (encoder);
    \draw[->] (motor) -- (encoder);
    \draw[-] (encoder) |- (point.west);
    \draw[->] (point) -| (dspace);
    \draw[->] (motor) |- (scope);
    \draw[->] (motor) |- (multi);


  %\node[label=above:C] (C)  [below right=0.7cm and 4cm of B1]
   %    {($2m-1$)};
    %\draw [draw,->] (input) -- node {$r$} (sum);
    %\draw [->] (sum) -- node {$e$} (controller);
    %\draw [->] (system) -- node [name=y] {$y$}(output);
    %\draw [->] (y) |- (measurements);
\end{tikzpicture}

%\end{document}

